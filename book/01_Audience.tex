\chapter{Define your Audience(s)}

Define your audience.

Before you write your first word of documentation, you need to know who
your audience is.

Unfortunately, by the time most of us got involved, there was already
hundreds of pages of documentation written to the wrong audience.
Fortunately, it's not an insurmountable problem, and most documentation
can be refocused on the right audience.

In technical documentation, there are three main audiences. This isn't
absolute, and many people have different ideas of the possible
audiences. Chances are that in your particular world, you'll end up
defining different subsets of your audience. However, go with me for a
moment, so that you can benefit from the exercise.

The first audience is the executives, or, at least, people who need to
know what your product does, and nothing else. Perhaps you're being
interviewed for a podcast, or you're developing a marketing brochure.
This is where you tell us what features you have, and perhaps throw in a
couple case studies. Any technical detail here is a liability. You can
even be forgiven for using a few poorly-defined buzzwords. The people
who know what they mean won't be reading this part.

The the other two audiences are less clearly defined, and there's a lot
of overlap between then.

There's the end-user, or the customer, as I prefer to call them. These
are the people who use the features of the software, need to know which
buttons to click, and don't care very much about what lies under the
hood. Except when they absolutely have to. In which case they want to
know solutions, rather then detailed descriptions of how the gears mesh.

And then there's the developers, who want to know all about the gears,
how to take them off and put them back on, how to replace them with
their own custom gears, and where to attach their own machine. There's
also a class of developers who want to know how to take what you've done
and make it look different. They want to know that you've implemented a
themeing engine that is easy to work with, and they want examples.
Although this latter subset thinks of themselves as designers, they are
in fact developers in the sense that they want to work with the code.
Or, if they're lucky, and you've provided a decent API, how to work with
that abstraction on top of the code.

These two major groups - the customers and the developers - have a large
gray area between then, and people are constantly moving back and forth
between the two groups.

You need to write to these people, and tell them all that they want to
know. You also want to avoid telling them a lot that they don't want to
know, because, to them, that's noise, and gets in the way of performing
the task that they actually care about.

Defining your audience frames everything else. If you get this wrong,
then all of your hard work, however well crafted it is, will be just so
much singing in a hurricane. You can be Pavarotti, but nobody will hear
you.

On the other hand, getting the audience precisely defined will enhance
the value of any effort you make. If you answer the right questions even
a little well, it's infinitely more valuable than answering the wrong
question to 25 decimal place accuracy.

The most common scenario in Open Source documentation is that
documentation is written for developers. The customers are given a nod
here and there, but we're primarily writing for the developers. We
pretend that because we write a theme API, and we write docs for
designers, that our docs are useful to customers. They're not. The
marketing brochure is considered far beneath our dignity.

In any given project, one of these audiences is going to be more
important, and your docs will reflect this. That's ok. But as your
project matures, your audience will grow and diversify, and your
documentation will need to mature to reflect this reality. Telling
people to RTFM may indeed be ok when you have a tiny community and they
are all hardened developers. It's not ok when you have professionals who
want to use your product to do their daily tasks, and are asking how to
tweak the wibble setting on the widget. They actually just want to do
that. You should believe them when they say that they really just want
to do that. They don't care how the widget spins. They don't even care
how to tweak the gronk setting. They just want to tweak the wibble
setting, and they expect you to tell them, and then go away.



