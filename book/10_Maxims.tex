\chapter{Maxims}

This isn't a chapter. This is a list of maxims. Each of these maxims
should become a chapter. Please feel free to grab one of them that
strikes your fancy, and write about it. Or add one.

\section{Find out who your audience is, and write to them. Hint: You are
not your audience.}

I think that this is chapter 1. If you don't know who your audience is,
you're going to be writing to yourself, and you probably don't know what
you need to know.

\section{Don't assume that you're audience knows anything. Also, don't assume
that they are stupid.}

Don't use the following words or phrases. Ever.
\begin{enumerate}
\item Trivial
\item Simple or simply
\item Of course
\item Self-evident
\item Self-documenting
\item As any fool can see
\item Obviously
\end{enumerate}

\section{Anticipate misconceptions, and gently correct}

\section{Strong leadership from an opinionated writer}

\section{Encourage customers to upgrade by showing them the advantages}

In the 1.3 docs, we should be telling folks ``this is easier in the 2.2
version''. In the 2.2 docs we usually shouldn't say ``here's how you
used to do it in version 1.2.17''.

\section{One working example teaches more than a page of theory}

\section{Pick a voice}
\begin{enumerate}
\item Don't switch from I to we
\item Decide who the listener is, and address them accordingly
\item Don't be terrified of the passive voice when describing software
\item Don't use colloquialisms or profanity
\item Be light and conversational
\end{enumerate}

\section{Lower the bar}

Make it as easy as possible for people to contribute corrections or
additions to the documentation:
\begin{enumerate}
\item Consider a wiki
\item Accept stuff in strange formats
\item Don't be defensive
\item Publically acknowledge submissions
\item Make it very clear who owns contributions
\item Test every example. Twice.
\end{enumerate}


